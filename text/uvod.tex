\chapter*{Úvod}
\phantomsection
\addcontentsline{toc}{chapter}{Úvod}

Kruhové podpisy jsou typy digitálních podpisů, které zajišťují anonymitu člena podepisující skupiny a zároveň potvrzují, že podpis je opravdu od člena dané skupiny. Rychlost rozšifrování zprávy a odhalení původce zprávy ze skupiny záleží na použitých protokolech k~podepisování. Tento typ podepisování se často používá u~kryptoměn (např. ShadowCash, Monero) k~ověřování plateb a zároveň anonymitě člověka, který platí.

S vývojem kvantových počítačů je potřeba začít vylepšovat současnou kryptografii, která není odolná vůči kvantovým počítačům. Jedním z~kvantově odolných kryptografických primitiv je kryptografie založená na mřížkách, která je popsána v~teoretické části. Toto kryptografické primitivum společně s~kruhovými podpisy jsme společně zkoumali jak fungují a následně implementovali řešení, které je popsáno v~praktické části.
