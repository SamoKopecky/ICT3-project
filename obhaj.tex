% Soubory musí být v kódování, které je nastaveno v příkazu \usepackage[...]{inputenc}

\documentclass[%        Základní nastavení
  %draft,    				  % Testovací překlad
  12pt,       				% Velikost základního písma je 12 bodů
	t,                  % obsah slajdů bude vždy začínat od shora (nebude vertikálně centrovaný)
	aspectratio=1610,   % poměr stran bude 16:10 (všechny projektory v učebnách na Technické 12 Brno),
	                    % další volby jsou 43, 149, 169, 54, 32.
	unicode,						% Záložky a informace budou v kódování unicode
]{beamer}				    	% Dokument třídy 'zpráva', vhodná pro sazbu závěrečných prací s kapitolami
%\usepackage{etex}

\usepackage[utf8]		  % Kódování zdrojových souborů je v UTF-8
	{inputenc}					% Balíček pro nastavení kódování zdrojových souborů
	
\usepackage{graphicx} % Balíček 'graphicx' pro vkládání obrázků
											% Nutné pro vložení logotypů školy a fakulty

\usepackage[          % Balíček 'acronym' pro sazby zkratek a symbolů
	nohyperlinks				% Nebudou tvořeny hypertextové odkazy do seznamu zkratek
]{acronym}						
											% Nutné pro použití prostředí 'acronym' balíčku 'thesis'

%% Balíček hyperref je volán třídou beamer automaticky, proto není třeba následujícího kódu:
%\usepackage[
%	breaklinks=true,		% Hypertextové odkazy mohou obsahovat zalomení řádku
%	hypertexnames=false % Názvy hypertextových odkazů budou tvořeny
%											% nezávisle na názvech TeXu
%]{hyperref}						% Balíček 'hyperref' pro sazbu hypertextových odkazů
%											% Nutné pro použití příkazu 'nastavenipdf' balíčku 'thesis'

\usepackage{cmap} 		% Balíček cmap zajišťuje, že PDF vytvořené `pdflatexem' je
											% plně "prohledávatelné" a "kopírovatelné"

%\usepackage{upgreek}	% Balíček pro sazbu stojatých řeckých písmem
											%% např. stojaté pí: \uppi
											%% např. stojaté mí: \upmu (použitelné třeba v mikrometrech)
											%% pozor, grafická nekompatibilita s fonty typu Computer Modern!

%\usepackage{amsmath} %balíček pro sabu náročnější matematiky

\usepackage{booktabs} % Balíček, který umožňuje v tabulce používat
                      % příkazy \toprule, \midrule, \bottomrule


%%%%%%%%%%%%%%%%%%%%%%%%%%%%%%%%%%%%%%%%%%%%%%%%%%%%%%%%%%%%%%%%%
%%%%%%      Definice informací o dokumentu             %%%%%%%%%%
%%%%%%%%%%%%%%%%%%%%%%%%%%%%%%%%%%%%%%%%%%%%%%%%%%%%%%%%%%%%%%%%%

\input{nastaveni}      % v tomto souboru doplňte údaje o sobě, o názvu práce...
                       % (tento soubor je sdílený s textem práce)

%%%%%%%%%%%%%%%%%%%%%%%%%%%%%%%%%%%%%%%%%%%%%%%%%%%%%%%%%%%%%%%%%%%%%%%%

%%%%%%%%%%%%%%%%%%%%%%%%%%%%%%%%%%%%%%%%%%%%%%%%%%%%%%%%%%%%%%%%%%%%%%%%
%%%%%%     Nastavení polí ve Vlastnostech dokumentu PDF      %%%%%%%%%%%
%%%%%%%%%%%%%%%%%%%%%%%%%%%%%%%%%%%%%%%%%%%%%%%%%%%%%%%%%%%%%%%%%%%%%%%%
%% Při vloženém balíčku 'hyperref' lze použít příkaz '\pdfsettings'
\pdfsettings
%  Nastavení polí je možné provést také ručně příkazem:
%\hypersetup{
%  pdftitle={Název studentské práce},    	% Pole 'Document Title'
%  pdfauthor={Autor studenstké práce},   	% Pole 'Author'
%  pdfsubject={Typ práce}, 						  	% Pole 'Subject'
%  pdfkeywords={Klíčová slova}           	% Pole 'Keywords'
%}
\hypersetup{pdfpagemode=FullScreen}       % otevření rovnou v režimu celé obrazovky
%%%%%%%%%%%%%%%%%%%%%%%%%%%%%%%%%%%%%%%%%%%%%%%%%%%%%%%%%%%%%%%%%%%%%%%

\usetheme{VUT} 				% barvy a rozložení prezentace odpovídající VUT FEKT
% alternativně lze použít jiná berevná témata, ale bez záruky. Například: 
%\usetheme{Darmstadt} \usecolortheme{default2}
\logoheader					% vytvoření zkráceného loga VUT FEKT v hlavičce slajdu, nechte odkomentované



\begin{document}

% v případě zakomentování následujícího se zobrazí v pravém dolním rohu slajdů klikatelné navigační symboly 
\disablenavigationsymbols

% titulní snímek, vysazen bez horních, dolních a postranních lišt (volba plain),
% není tak vysazen ani nadpis snímku
\maketitle

%%%%%%%%%%%%%%%%%%%%%%%%%%%%%%%%%%%%%%%%%%%%%%%%%%%%%%%%%%%%%%%%%%%%%%%
% 1. snímek s cíli (zadaním) práce
\begin{frame}[c]
    \large{
	% nadpis snímku
	\frametitle{Cíle práce}
    \begin{itemize}
        \item Seznámení s kruhovými podpisy
        \item Vybrání postkvantového algoritmu
        \item Implementace postkvantového schématu kruhových podpisů
    \end{itemize}
}
\end{frame}

\begin{frame} 
	% nadpis snímku
	\frametitle{Kruhový podpis}
    \large{
    \begin{itemize}
        \item Publikován v~roce 2001
        \item Anonymita
        \item Každý člen má vlastní pár asymetrických klíčů
    \end{itemize}
    }
	\begin{figure}[htbp]
  \centering
  \includegraphics[width=0.45\textwidth]{img/ring_signature.pdf}
  
  
\end{figure}
\end{frame}


\begin{frame}
	% nadpis snímku
	\frametitle{Post kvantové algoritmy}
    \large{
    \begin{itemize}
        \item První kruhový podpis byl založený na algoritmu RSA
        \item Kvantové počítače jsou schopné prolomit současnou kryptografii
        \item Kryptografie založená na mřížkách
    \end{itemize}
    \begin{figure}[htbp]
      \centering
      \includegraphics[width=0.4\textwidth]{img/mrizky.png}
      
  
    \end{figure}
    }	
\end{frame}

\begin{frame}
	% nadpis snímku
	\frametitle{Protokol L2RS}
    \large{
    \begin{itemize}
        \item Implementace v Python
        % \item Jediná využívaná knihovna je \texttt{numpy}
        \item Založený na BLISS postkvantovém podpisu
        \item Vytváření podpisu trvá průměrně 43\,ms
        \item Ověření podpisu trvá průměrně 38\,ms
        \item Velikost podpisu roste lineárně 
    \end{itemize}
    }	
    \begin{table}[htbp]
  \centering

  \begin{tabular}{|l|c|r|}
    \hline
    Typ              & Počet uživatelů & Velikost (B) \\
    \hline
    Veřejný klíč     & -               & 896         \\
    Privátní klíč    & -               & 4 480        \\
    Podpis           & 1               & 7 168        \\
    Podpis           & 2               & 12 544       \\
    \hline
  \end{tabular}
  \label{sizes}
\end{table}

\end{frame}


\begin{frame}[c] 
    
	\frametitle{Praktická ukázka}
	\begin{center}
		\begin{figure}[htbp]
  \centering
  \includegraphics[width=0.65\textwidth]{img/network_diagram.pdf}
  
  
\end{figure}
	\end{center}
\end{frame}

\begin{frame}[c] 

	\frametitle{Závěr}
	\large{
    \begin{itemize}
        \item Teoretické seznámeni s problematikou
        
        \item Úspěšná implementace zadání
        
        
    \end{itemize}
    }
\end{frame}


\begin{frame}[c] 

	\frametitle{\mbox{ }}
	\begin{center}
		{\Huge Děkujeme za pozornost!}
	\end{center}
\end{frame}



\end{document}
